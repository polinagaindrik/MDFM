\documentclass[10pt,A4paper]{article}
\usepackage[a4paper,
            inner=2.5cm,
            outer=3cm
            ]{geometry}
\usepackage{acronym}
\usepackage{amsmath,amssymb}
\usepackage[aboveskip=1pt,labelfont=bf,labelsep=period,justification=raggedright,singlelinecheck=off]{caption}
\usepackage{changepage}
\usepackage{cite}
\usepackage{nameref,hyperref}
\usepackage[right]{lineno}
\usepackage[nopatch=item]{microtype}

\usepackage{xcolor}
\usepackage{multirow}
\usepackage{ctable} % for \specialrule command
\usepackage{graphicx}
\usepackage{subcaption}
\usepackage{tikz}

\newcommand{\mbP}{\textbf{P: }}
\newcommand{\mbC}{\textbf{C: }}

\begin{document}
\tableofcontents
\section{General observations on bacterial diversity}
\subsection{Measurement methods}
\begin{itemize}
    \item NGS: diversity in the meat sample
    
    \item MiBi: bacterial count in a certain AGAR\\
        - PCA (Plate Count Agar): non-selective media, aerobic bacteria growth\\
        - MRS: selective media, lactic acid bacteria growth\\
        - VRBD: Enterobacteria

    \item MALDI-ToF: diversity in the AGAR

\end{itemize}

\subsection{Bacterial key players}
\textbf{From NGS data:}
\begin{itemize}
    \item Leuconostoc (Can measure in MiBi in MRS)
    \item Lactobacillus (In MRS, for higher temperatures)
    \item \textit{Lactococcus} (Not quantative through MiBi, Exp2, 10Grad)
    \item Brochothrix  (Measure in MiBi in PC)
    \item Carnobacterium (Measure in MiBi in PC)
    \item Pseudomonas (Noticable in 14 Grad in PC, but not in 2Grad, can be mistake in NGS?)
    \item Enterobacteriaceae (e.g. Serratia, Can measure in PC and/or VRBD)
\end{itemize}


\textbf{In MiBI + MALDI-ToF:}
\begin{itemize}
    \item Brochothric thermospacta (in PC)
    \item Carnibacterium divergens (LAB)(in PC)
    \item Leuconostoc gelidum (LAB) (in MRS)
    
    \item Cronobacter sp. (VRBD) (?)
    \item Serratia (VRBD) (?) (at high temperatures in MALDI but not present on NGS? In tables as well)
\end{itemize}

\subsection{Noticed trends}
\begin{itemize}
    \item \textbf{10 Grad, PC, O2:} Exp2 at the beginning much more Carnobacterium -> Brochothrix is inhibited\\
    In Exp 2 Brochothr in not growing --  at the end the maximum count value is smaller then in Exps with Broch.

    \item \textbf{2 Grad, PC, O2:} Similarly, at the end Exp 8 has higher value and the persentage of the Brochothrix is higher.\\
    Also the more diverse at the end Exp 1 has the smaller bacterial count. 

    \item \textbf{14 Grad, PC, O2:} New players are: Serratia (Enterobacteriaceae) and Pseudomonas in addition to Br. and Carn.
    \textbf{Assumption:} 14 Grad is less diverse. Why? LAB inhibit Brochothris, hence, diff initial concentrations Broch. and Carn. gives quite diff dynamics.\\
    At 14 Grad We see much more Pseudomonas. Pseudomonas inhibit LAB but does not affect Brochothrix.
    Hence, LAB affect less Brochothrix and we have less variation in the growth.
\end{itemize}

\begin{itemize}
    \item \textbf{2 Grad, NGS, O2:} We have a lot of LAB: Leuconostoc, Lactobacillus. But also a lot of Pseudomonas (which we do not see in MALDI, Exp7).
    Pseudomonas I can see a little bit in PC media. But they are not dominant there. (how to detect, not clear?) (Or meaybe it's measurement error in NGS due to sampling) \\
    (Can it be that Pseudomonas are better detected for NGS, due to their DNA sequence or whatever?)

    \item  \textbf{10 Grad, NGS, O2:} Lactococcus (Exp 2) and Pseudomonas (Exp 5). \\
    Lactococcus (a little bit in PC, a little bit in MRS < 10 percent, but we do not get really numbers for them).
    Carno (Exp 2, 4 percent), Brochothrix (Exp 5, 15 percent)

    \item \textbf{14 Grad, NGS, O2:} Pseudomonas (mostly) + Enterobacteria (PC/VRBD media then needed ?) (it is probably Serratia)
\end{itemize}

\newpage
\subsection{Ideas}
\textbf{Idea:}
\begin{itemize}
    \item \textbf{Assumption:} assume that bacteria seen in PCA do not interact with bacteria in media MRS. 
    They grow in different media and consume different resource. 
    \item Then we can build independent models for each media. Defining for each model critical count.
    Or on the count sum of the model results.
\end{itemize}
Then \textbf{PCA Model}:\\
\begin{itemize}
    \item 4 species compete for 1 common resurce: \\
          - \textbf{\textit{Brochothrix, Carnobacterium}:} \\
          (dominate at low temp 2 Grad, competition depending on the initial condition. More Broch. at the end gives higher nmax.
          It happens when Br. where initially more)\\
          - \textbf{\textit{Serratia} (corr. to Enterobact.):}\\
           (dominate at high temp 14 Grad)
           (At high temperatures may be also seen in VRBD?)\\
          - \textbf{\textit{Pseudomonas}:}\\
           (present always, more at 14 Grad, does not depend on T)
           (But data not consistent in NGS and MALDI. Can we really measure them?)\\
    \item From literature known:\\
          - \textit{LAB} (prob. \textit{Carnobact.}) inhibit growth of Brochothrix\\
          - \textit{Pseudomonas} and \textit{Enterobact (Serratia)} have no direct influence on the growth of \textit{Brochothrix} (prob diff resurce as well??).\\
          - \textit{Pseudomonas} and \textit{Enterobact (Serratia)} inhibit \textit{LAB (Carnobacterium)}\\
\end{itemize}

\textbf{MRS Model}:
\begin{itemize}
    \item 2 key players: \textbf{\textit{Leuconostoc}}, \textbf{\textit{Lactobaccilus}} \\
    \item 2 Grad: 100\% \textit{Leuconostoc}\\
          10 Grad: 70/30\% (\textit{Leuconostoc}/\textit{Lactobaccilus})\\
          14 Grad: 50/50 \% (or \textit{Lactobaccilus} can also dominate)
\end{itemize}

\textbf{VRBD media}: Values for Serratia at higher temperatures.\\

\textbf{But:}
\begin{itemize}
    \item Some bacteria present both in PC and MRS but in less amount.\\
    Then they have common nutritient? 
    \item Use NGS to determine which bacteria/media is more active in a certain case?
    \item The usage of the initial distributionin NGS. How it affects time dynamics? Search for patterns?
    \item Criteria for critical time? Sum of all bacteria or for each bacteria responsible of the spoilage?
    The time at which this bacteria was present in certain quantity?
\end{itemize}

\newpage
\section{Fusion model. Mathematical formulation.}
\subsection{Observables}
Assume initial / real-life system of $N$ bacterial species.
The bacterial count (colony forming units, cfu) of each bacterial species described with vector $\mathbf{x} = (x^1, ... x^N)^T$.
Observables of the systems obtained with 3 different methods: MiBi (standard microbiology), MALDI-TOF and NGS (diversity measurement variables).
The mathematical description of the variables:
\begin{itemize}
    \item \textbf{NGS:} Observable vector $(N\times1)$.\\
    Element of the observable vector: 
    \begin{equation}
        (y_\text{NGS})^n = \frac{x^n}{\sum_i x^i}
    \end{equation}
    Then the total observable NGS vector: 
    \begin{equation}
        \boxed{\mathbf{y}_\text{NGS} = \frac{1}{\sum_i x^i} \mathbf{x} = \frac{\mathbf{x}}{\lVert \mathbf{x} \rVert}}
    \end{equation}
    where $\mathbf{1}$ is unit vector of size $(N\times1)$.

    \item \textbf{MiBi:} Observable vector $(M\times1)$, where $M$ is the number of different media used for MiBi measurement.\\
    To get the MiBi observables the sample is planted to a specific media either selective (e.g. MRS) or allowing the majority of bacteria to grow (e.g. PC).
    Hence, bacteria is subjected to some sort of media filtering described by function $\mathbf{f}(\mathbf{x})$. 
    Then the element of the observable vector:
    \begin{equation}
        (y_\text{MiBi})^m = \sum_i f^m_i
    \end{equation}
    Where $f^m_i$ is an element of the matrix $F (M \times N)$ describing the number of bacteria grown in media.
    The MiBi observable vector is then will be:
    \begin{equation}
        \mathbf{y}_\text{MiBi} = \sum_i \begin{pmatrix}
            f^1_i \\ \vdots \\ f^M_i
        \end{pmatrix} = F \mathbf{1}
    \end{equation}
    If we assume that media filtering is a linear function of bacteria in meat: $f^m_i=s^m_i \cdot x^i$.\\
    Then the element of the observable vector is
    \begin{equation}
        (y_\text{MiBi})^m = \sum_i s^m_i x^i
    \end{equation}
    And the observable vector is described as:
    \begin{equation}
        \boxed{\mathbf{y}_\text{MiBi} = S \mathbf{x}}
    \end{equation}
    Where $S$ is a $(M \times N)$ matrix whose element $s^m_n$ describes the probability of the colony forming of the species $n$ in media $m$.

    \item \textbf{MALDI:}  Observable vector $(M \times N)$.
    Element of the observable vector:
    \begin{equation}
        (y_\text{MALDI})^m_n = \frac{f^m_n}{\sum_i f^m_i} = \frac{s^m_n  x^n}{\sum_i s^m_i  x^i}
    \end{equation}
    Then the observable vector:
    \begin{equation}
        \mathbf{y}_\text{MALDI} = \begin{pmatrix}
            \frac{1}{\sum_i f_i^1} & 0                      & ... & 0           \\
            0                      & \frac{1}{\sum_i f_i^2} & ... & 0           \\
            \vdots                 & \vdots                 &     & \vdots      \\
            0                      & 0                      &     & \frac{1}{\sum_i f_i^M}
        \end{pmatrix} F
        \end{equation}
        \begin{equation}
            \boxed{\mathbf{y}_\text{MALDI} = \begin{pmatrix}
                \frac{1}{\sum_i s_i^1 x^i} & 0                           & ... & 0 \\
                0                          & \frac{1}{\sum_i s_i^2  x^i} & ... & 0 \\
                \vdots                     & \vdots                      &     &\vdots \\
                0                          & 0                           &     &\frac{1}{\sum_i s_i^M  x^i}
        \end{pmatrix} 
        \begin{pmatrix}
            s_1^1 x^1 & ... & s_n^1 x^N \\
            \vdots    &     & \vdots    \\
            s_1^M x^1 & ... & s_n^M x^N 
        \end{pmatrix}} 
    \end{equation}
\end{itemize}

\subsection{Clustering constraints}
We group $N$ bacteria ($x^n$) into $K$ different clusters: $C^k$.
Then the clusters can be described as:
\begin{equation}
    C^k = \sum_n x^n p_n^k, \hspace{1cm} p_n^k = \begin{cases}
        0, \text{ if bacterial species $n$ belongs to cluster } k \\
        1, \text{ if bacterial species $n$ doesn't belong to cluster } k
    \end{cases}
\end{equation}
Or in matrix form: 
\begin{equation}
    \mathbf{C} = P \mathbf{x}
\end{equation}
where $\mathbf{C}$ is $(K \times 1)$ vector of cluster values.
$R$ is the $(K \times N)$ transition matrix from bacterial species to bacterial clusters with elements $p_n^k$ corresponding to species $n$ and cluster $k$.
The built clusters should be still able to reproduce the observables as MiBi and MALDI.\\
E.g., constraint due to MiBi observable:
\begin{equation}
    \sum_i s_i^m x^i = \sum_k C^k \tilde{s^m_k} = \sum_k  \tilde{s^m_k} \sum_i x^i p_i^k = \sum_i x^i \sum_k \tilde{s^m_k} p_i^k
\end{equation}
should hold for any vector $\mathbf{x}$.
That follows:
\begin{equation}
    \boxed{s_i^m = \sum_k \tilde{s^m_k} p_i^k}
\end{equation}
And in matrix form constraint on clustering procedure due to MiBi measurements:
\begin{equation}
    \boxed{S = \tilde{S} P}, \hspace{1cm} \text{or} \hspace{1cm}  \boxed{P = \tilde{S}^{-1} S}
\end{equation}
Check dimensionality: $(M \times N) = (M \times K) \cdot (K \times N)$.

\subsection{Clustering interpretation}

\begin{tikzpicture}
    \begin{axis}[option of the axis]
        \draw[help lines, color=gray!30, dashed] (-0.1,-0.1) grid (4.9,4.9);
        \draw[->,ultra thick] (0,0)--(5,0) node[right]{$x_1$};
        \draw[->,ultra thick] (0,0)--(0,5) node[above]{$x_2$};
        \draw[-,thick]  (0, 0)--(5,5) node[above]{$\mathbf{C}$};
        \draw[->,thick] (0, 0)--(3.75, 3.75) node[above]{};
        \draw[-,thick]  (3, 4.5)--(3.75, 3.75) node[above]{};
        \draw[->,thick] (0, 0)--(3, 4.5) node[above]{$\mathbf{x}$};
        \draw[->,thick] (0, 0)--(4, 2.5) node[above]{$S\mathbf{x}$};
        \draw[-,thick]  (3.3, 3.3)--(4, 2.5) node[above]{};
    \end{axis}
\end{tikzpicture}

To measure NGS data vector:
\begin{equation}
    \mathbf{x} = (x_1, ..., x_N)
\end{equation}
For MALDI-TOF after media filtering:
\begin{equation}
    S\mathbf{x} = (s^m_1 x_1, ..., s^m_N x_N)
\end{equation}

The new cluster vector:
\begin{equation}
    \mathbf{C} = <\mathbf{x} \cdot \mathbf{e_C}> \mathbf{e_C} = (C^1, ..., C^K)
\end{equation}

Cluster vector after media filtering:
\begin{equation}
   \tilde{S} \mathbf{C} = <S \mathbf{x} \cdot \mathbf{e_C}> \mathbf{e_C} = (s^m_1 C^1, ..., s^m_K C^K)
\end{equation}

Recalculating $x$ variables into cluster variables $C$:
\begin{equation}
    (y^C_\text{ngs})^k = \sum_i^N p_i^k (y_\text{ngs})^i = \frac{\sum_k^K p_i^k x^i}{\sum_j^N x^j} = \frac{C^k}{\sum_l^K C^l}
\end{equation}

\begin{equation}
    (y^C_\text{maldi})_k^m = \sum_i^N p_i^k (y_\text{maldi})_i^m = \frac{\sum_k^K p_i^k s^m_i x^i}{\sum_j^N s^m_j x_j} = \frac{\tilde s^m_k C^k}{\sum_l^K  \tilde s^m_l C^l}
\end{equation}

\begin{equation}
    (y^C_\text{mibi})^m = (y_\text{mibi})^m = \sum_j^N s^m_j x_j = \sum_l^K  \tilde s^m_l C^l
\end{equation}


\subsection{Estimation of the media filtering matrix}

Let us define the filtering matrix of the media $m$ for the system consisting of $N$ different bacteria types:
\begin{equation}
    S^m_\text{filt} = \begin{pmatrix}
        s^m_1  & 0       & ...    & 0      \\
        0      & s^m_2   & ...    & 0      \\
        \vdots & \vdots  & \vdots & \vdots \\
        0      & 0       & ...    & s^m_N  \\
    \end{pmatrix}
\end{equation}
$S^m_\text{filt}$ is a diagonal matrix whose diagonal elements $s^m_i$ define probability of growing of the  bacteria $i$ in the media $m$.
Remember that the vector of the real bacteria counts in meat $\mathbf{x}$ is a $( N \times 1)$ vector:
\begin{equation}
    \mathbf{x} = \begin{pmatrix} x^1 \\ x^2 \\ \vdots \\ x^N
    \end{pmatrix}
\end{equation}

And the NGS observable:
\begin{equation}
    \mathbf{y}_\text{NGS} = \frac{\mathbf{x}}{\lVert \mathbf{x} \rVert} = \frac{\mathbf{x}}{\sum_j x^j}
    \label{eq:obs_ngs}
\end{equation}

The bacterial count that we observe in media is defined by:
\begin{equation}
    \mathbf{x}_\text{filt} = S^m_\text{filt} \mathbf{x} = \begin{pmatrix} s^m_1 x^1 \\ s^m_2 x^2 \\ \vdots \\ s^m_N x^N \end{pmatrix}   
\end{equation}

Then the MALDI-ToF observable in media $m$ is:
\begin{equation}
    \mathbf{y}^m_\text{MALDI} = \frac{\mathbf{x}_\text{filt}}{\lVert \mathbf{x}_\text{filt} \rVert} = \frac{ S^m_\text{filt} \mathbf{x}}{\lVert S^m_\text{filt} \mathbf{x} \rVert}
    \label{eq:obs_maldi0}
\end{equation}
Multiplying \ref{eq:obs_ngs} to $S^m_\text{filt}$ and using \ref{eq:obs_maldi0} we can define the relation between MALDI-ToF and NGS observables:
\begin{equation}
    \mathbf{y}^m_\text{MALDI} = \underbrace{\frac{\sum_j x^j}{\sum_j^N s^m_j x^j}}_{A(t)} S^m_\text{filt} \mathbf{y}_\text{NGS}
\end{equation}

To define time dependent constant $A(t)$ we will take into account that $\mathbf{y}^m_\text{MALDI}$ defines proportions of bacteria in media and the sum of all bacteria in the sample should be equal 1:
\begin{equation}
    \sum_i^N (\mathbf{y}^m_\text{MALDI})_i = \sum_i^N s_i^m (\mathbf{y}_\text{NGS})_i A(t) = 1
 \end{equation}
 \begin{equation}
    A(t) = \frac{1}{\sum_i^N s_i^m (\mathbf{y}_\text{NGS})_i} = \frac{1}{\lVert S^m_\text{filt} \mathbf{y}_\text{NGS} \rVert}
 \end{equation}
 \begin{equation}
    \boxed{\mathbf{y}^m_\text{MALDI} = \frac{S^m_\text{filt} \mathbf{y}_\text{NGS}}{\lVert S^m_\text{filt} \mathbf{y}_\text{NGS} \rVert}}
    \label{eq:obs_maldi}
\end{equation}

To calculate the real bacterial values $x^i(t)$ we need to know scaling factor $\sum_j x^j$.
Note that the MALDI-ToF scaling factor $A(t)$:
\begin{equation}
    A(t) = \frac{\sum_j x^j}{\sum_j^N s^m_j x^j} = \frac{1}{\lVert S^m_\text{filt} \mathbf{y}_\text{NGS} \rVert}
\end{equation}
Using that MiBi observables are known and y definition for media $m$:
\begin{equation}
    (y_\text{mibi})^m = \sum_j^N s^m_j x^j = \lVert S^m_\text{filt} \mathbf{x} \rVert
\end{equation}
We get that:
\begin{equation}
    \sum_j x^j = \frac{y_\text{mibi}^m}{\lVert S^m_\text{filt} \mathbf{y}_\text{NGS} \rVert}
\end{equation}
And the real bacterial values without media filtering are:
\begin{equation}
    \boxed{\mathbf{x}_\text{data} = \mathbf{y}_\text{NGS} \sum_j x^j = \frac{\mathbf{y}_\text{NGS} y_\text{mibi}^m}{\lVert S^m_\text{filt} \mathbf{y}_\text{NGS} \rVert}}
    \label{eq:X_real}
\end{equation}

Clustered measurement points are then:
\begin{equation}
    \mathbf{C}_\text{data} = P \mathbf{x}_\text{data}
\end{equation} 
The cost function is 
\begin{equation}
    ll = \frac{(\mathbf{C}_\text{data} - \mathbf{C}_\text{model})^2}{2 \sigma^2} \propto \bigg (P \frac{\mathbf{y}_\text{NGS} y_\text{mibi}^m}{\lVert S^m_\text{filt} \mathbf{y}_\text{NGS} \rVert} - \mathbf{C}_\text{model} (t, p) \bigg )^2
\end{equation}

So we estimate parameters $p$ and get model prediction $ \mathbf{C}_\text{model} (t, p)$

\subsection{Estimation of the media filtering matrix (New)}
To account for data (species) measured with MALDI-ToF method but not captured with NGS measurement approach we introduce matrix:
\begin{equation}
    T = \begin{pmatrix}
        t_1    & 0      & ... & 0      \\
        0      & t_2    & ... & 0      \\
        \vdots & \vdots & ... & \vdots \\
        0      & 0      & ... & t_N    \\
    \end{pmatrix}
\end{equation}
where
\begin{equation}
    t_i = \begin{cases}
        0 \text{ if NGS cannot measure bacteria $i$, nowhere in data present } \\
        1 \text{ if at least 1 NGS measurement of bacteria $i$ not equal to zero } \\
    \end{cases}
\end{equation}

Then we define NGS observable as
\begin{equation}
    \mathbf{y}_\text{NGS} = \frac{T \mathbf{x}}{\lVert T \mathbf{x} \rVert}
\end{equation}
And MALDI-ToF observable for media $m$ is
\begin{equation}
    \mathbf{y}^m_\text{MALDI} = \frac{S^m \mathbf{x}}{\lVert S^m \mathbf{x} \rVert}
\end{equation}
Multiplying MALDI observables with $T$ matrix and using that $ST=TS$ as diagonal matrices:
\begin{equation}
    T \mathbf{y}^m_\text{MALDI} = \frac{TS^m \mathbf{x}}{\lVert S^m \mathbf{x} \rVert} = \frac{S^m T \mathbf{x}}{\lVert S^m \mathbf{x} \rVert} =
    S^m \frac{T \mathbf{x}}{\lVert T \mathbf{x} \rVert} \frac{\lVert T \mathbf{x} \rVert}{\lVert S^m \mathbf{x} \rVert} = S^m \mathbf{y}_\text{NGS} \underbrace{\frac{\lVert T \mathbf{x} \rVert}{\lVert S^m \mathbf{x} \rVert}}_{A(t)}
\end{equation}

To calculate constant $A(t)$ we normalize the equation above:
\begin{equation}
    A(t) = \frac{\lVert T \mathbf{y}^m_\text{MALDI} \rVert}{\lVert S^m \mathbf{y}_\text{NGS} \rVert}
\end{equation}
where norms are calculated from data.
Then the final relation is 
\begin{equation}
    T \mathbf{y}^m_\text{MALDI} = S^m \mathbf{y}_\text{NGS} \frac{\lVert T \mathbf{y}^m_\text{MALDI} \rVert}{\lVert S^m \mathbf{y}_\text{NGS} \rVert}
\end{equation}

And cost function term:

\begin{equation}\boxed{
    cost1 = \frac{1}{N_{meas}} \Big \lVert T \mathbf{y}^m_\text{MALDI} \lVert S^m \mathbf{y}_\text{NGS} \rVert -  S^m \mathbf{y}_\text{NGS} \lVert T \mathbf{y}^m_\text{MALDI} \rVert \Big \rVert _2}
\end{equation}
Where norm is the squared sum over all measurement data (all experiments and all timepoints) and over all bacteria species.
In case of $T = 1$ is unity matrix:
\begin{equation}
    \mathbf{y}^m_\text{MALDI} = S^m \mathbf{y}_\text{NGS} \frac{1}{\lVert S^m \mathbf{y}_\text{NGS} \rVert}
\end{equation}

$S^m$ matrices for each media $m$ are determined till scaling constant (one degree of freedom pro media).
To correctly build observables from the same model $x$ we need to reduce it to one degree of freedom pro system.
Assuming that the real $S$ matrix for each media $m_k$ can be written with scaling constant $s_0$ as
\begin{equation}
    \tilde S^{m_k} = s_0^{m_i} S^{m_k}
\end{equation}

Then the ratio between MALDI-ToF measurements in two different media $m_i$ and $m_j$:
\begin{equation}
    \frac{\mathbf{y}^{m_k}_\text{MALDI}}{\mathbf{y}^{m_l}_\text{MALDI}} = \frac{ \tilde S^{m_k} \mathbf{x}}{ \tilde S^{m_l} \mathbf{x}} \frac{\lVert S^{m_l} \mathbf{x} \rVert}{\lVert S^{m_k} \mathbf{x} \rVert} =
    \frac{s^{m_k}_0 S^{m_k} }{s^{m_l} S^{m_l}} \frac{y^{m_l}_\text{MiBi}}{y^{m_k}_\text{MiBi}} =
    s_0^{kl} \frac{S^{m_k} }{S^{m_l}} \frac{y^{m_l}_\text{MiBi}}{y^{m_k}_\text{MiBi}}
\end{equation}
allows to estimate the ratio between two scaling constants in different media $s_0^kl$.
The cost function to estimate scaling constants:

\begin{equation}\boxed{
    cost2 = \frac{1}{N_{meas2}} \Big \lVert S^{m_k} \mathbf{y}^{m_l}_\text{MALDI} y^{m_l}_\text{MiBi} 
     -  s_0^{kl} S^{m_l} \mathbf{y}^{m_k}_\text{MALDI} y^{m_k}_\text{MiBi}  \Big \rVert _2}
\end{equation}
where norm is the squared sum over all experiments, bacterial species and media $l$.

(?? Add setting s to zero if ti = 0)


\subsection{Projection matrix estimation}

\begin{equation}
    C = Px = \begin{pmatrix}
        p^1_1 & \dots & p^1_N  \\
        \vdots &      & \vdots \\
        p^K_1 & \dots & p^K_N  \\
    \end{pmatrix} x
\end{equation}
Here $P$ is $(K \times N)$ projection matrix to a system with reduced number of variables $C$.
\begin{equation}
    x = P_\text{inv} C
\end{equation}
where $P_\text{inv}$ is a pseudo-inverse matrix of $P$.

To get the projection matrix $P$ we minimize the following cost function:
\begin{equation}
    Cost = \lVert P_\text{inv} P \mathbf{y}_\text{NGS} - \mathbf{y}_\text{NGS} \rVert^2 +
           \lVert S P_\text{inv} P S^{-1} \mathbf{y}_\text{MALDI} - \mathbf{y}_\text{MALDI} \rVert^2 +
           \underbrace{\lVert P P^T - I \rVert^2}_\text{ortogonality} +
           \underbrace{\lVert \lVert p^k \rVert^2  - 1\rVert^2}_\text{normalization}
\end{equation}
where $I$ is identity matrix and $p^k = (p^k_1 \dots p^k_N)$

\section{Justification for the projection matrix}
Considering projection of the full system from $N$ to $m$ dimensions

$P: N \rightarrow m$

Number of parameters included in projection matrix $P$ is $(N-m)m$.\\
Number of the parameters of the full system of $N$ dimensions is  $N(N-1) + 2 n_{st} N + 1$.\\
Number of the parameters of the reduced system of $m$ dimensions is  $m(m-1) + 2 n_{st} m + 1$.\\

Then the projection makes sense if:
\begin{equation}
    (N-m)m + N(N-1) + 2 n_{st} N + 1 < m(m-1) + 2 n_{st} m + 1
\end{equation}
\begin{equation}
    N(N + 2 n_{st} - 1) < m(N + 2 n_{st} - 1)
\end{equation}
which holds always when 
\begin{equation}
    m < N.
\end{equation}

\section{Parameter Estimation with Reduced Model}
 
The optimizer tries to find a set of parameters $\mathbf{p}$ for a "small" model that minimizes the difference between the real data and the model observables using log-likelihood function.
One iteration step to calculate log-likelihood is the following:
\begin{enumerate}
    \item Pick parameter set $\mathbf{p}$ for a "small" model.
    \item Solve "small" model solution for the chosen parameter set $\mathbf{C}(t, \mathbf{p})$.
    \item Back-project the solution on the "big" model using pseudoinverse projection matrix for estimated before projection $P$.
    \begin{equation}
        \mathbf{x} (t) = P^{-1} \mathbf{C} (t)
    \end{equation}
    \item Calculate observables for the "big" model $\mathbf{y}_\text{MiBi}$,  $\mathbf{y}_\text{MALDI}$,  $\mathbf{y}_\text{NGS}$.
    \item Calculate cost function as a sum over all measurement data and corresponding model observables:
    \begin{equation}
        \text{cost} = \sum_i \frac{(y_i^\text{data} - y_i^\text{model})^2}{2 \sigma_i^2}
    \end{equation}
\end{enumerate}
The optimizing function aims to find the parameters $\mathbf{p}$ that correspond to the global minimum of the cost function.

\section{Overview of the model}
\begin{enumerate}
    \item Daten: \\
        NGS:  $ \mathbf{y}_\text{NGS} = \frac{\mathbf{x}}{\lVert \mathbf{x} \rVert}$\\
        MiBi: $ \mathbf{x}_\text{filt} = {\lVert S^m \mathbf{x} \rVert}$\\
        MALDI-ToF in media $m$: $ \mathbf{y}^m_\text{MALDI} = \frac{ S^m \mathbf{x}}{\lVert S^m \mathbf{x} \rVert}$\\
    \item Model Possibilities:
    \begin{enumerate}
        \item 1-species model for each temperature and media (need MiBi data)
        \item \textbf{1-species model for each media} (temperature-dependent model, need MiBi data)
        \item Multi-species model for each media (temperature-dependent model, more complicated model for smaller uncertinties need Mibi/ MALDI)
        \item \textbf{Multi-species model} (temperature- and media- dependent model, more complicated model for smaller uncertinties need MiBi/MALDI/NGS)
        NGS allows us to determine S matrices and, hence, to connect data for different media.
    \end{enumerate}
    (c) and (d) use diversity data and need dimensionality reduction.

    \item Dimensionality Reduction:
    \begin{itemize}
        \item Projection matrix $Px$: Due to L1-norm division cannot build observables with reduced coordinates.
        \item Need to back-project $P^{-1} P x$ - Losing information.
        \item Information loss vs. simplicity/computational speed (number of clusters)
        \item Large number of clusters (less information loss) and more complicated ODE model vs. Small number of clusters (loss information is higher and model is simple to solve).
    \end{itemize}
    \item Temperature-dependence
    \begin{enumerate}
        \item P: temperature-dependent or one P for all temperatures
        \item ODE model: polynomial T-dependence of parameters (or some function) or specific parameter for each temperature
        \item Till now: Different P for each temperature and linear T-dependence of the parameters - Give a problem.
        \item What to do?\\
        \textbf{1. P- common, ODE - linear of T}\\
                2. P(T), but ode params calculated for each temperature?\\
                3. P(T), ode params non-linear with  T ???  Can we even determine this dependence?
                If we have P(T) as a continuous function then probably ODE parameters can also be continuous function of T.
        \item 1. Option works good in-silico for both 2 and 6 clusters. But doesn't work on real data.
    \end{enumerate}
    \item Other alternatives - ?
\end{enumerate}
\end{document}